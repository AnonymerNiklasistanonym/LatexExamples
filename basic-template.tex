% Document type/layout
% ==========================================================================
\documentclass{article}				% -> Article layout (default)
%\documentclass[journal]{IEEEtran}	% -> Two column journal layout
%\documentclass{report}				% -> Report layout
% ==========================================================================

% Add packages to expand the functionality with the following command:
% \usepackage[options]{packagename}
% ==========================================================================
\usepackage{bookmark}		% - Show all sections in the pdf as metadata
\usepackage{hyperref}		% - For metada rendering in PDFs
\usepackage{lastpage}		% - To get the toatal numbe rof pages
\usepackage{fancyhdr}		% - For header and footer
\usepackage{tocloft}		% - For table of contents
\usepackage{xcolor}			% - Use colors
\usepackage{graphicx}		% - Load/Display graphics
\usepackage{hyperref}		% - Insert hyperlinks
\usepackage{enumitem}		% - Needed for text lists
\usepackage{listings}		% - Display source code
\usepackage[english]{babel}	% - Umlauts (GERMAN PACKAGE: "ngerman")
\usepackage[utf8]{inputenc}	% - Umlauts (Input encoding - understand "ä")
\usepackage{lmodern}		% - Font encoding (find even "Hölleheiß" in PDF)
\usepackage{titlesec}		% - Go deeper than a subsubsection
\usepackage{dcolumn}		% - FOr decimal sorting in tables
\usepackage{multirow}		% - Needed for some tables
% ==========================================================================

% Variables
% ==========================================================================
\newcommand*{\mytitle}{The title}		% - Document title
\newcommand*{\myauthor}{One author}		% - Document author (one author)
% \newcommand*{\myauthorOne}{AuthorOne}	% - Document authors (many authors)
% \newcommand*{\myauthorTwo}{AuthorTwo}	% - Document authors (many authors)
% \newcommand*{\myauthor}{\texorpdfstring{\myauthorOne, \myauthorTwo}{\myauthorOne;\myauthorTwo}} % - Document authors (many authors)
\newcommand*{\mysubject}{The subject}			% - Document subject
\newcommand*{\mydate}{\normalsize 03.09.2018}	% - Document release date
% -------------------------------------------------------------------------
\definecolor{color-name}{HTML}{FF7F00}
\definecolor{orange}{RGB}{255,127,0}
\definecolor{orange}{HTML}{FF7F00}
% ==========================================================================

% Document data
% ==========================================================================
\title{\mytitle}	% - Set the visual title
\author{\myauthor}	% - Set the visual author
\date{\mydate}		% - Set the visual date
\hypersetup{		% - Metadata setup
	pdftitle={\mytitle},		% - Set title of the PDF file
	pdfsubject={\mysubject},	% - Set subject of the PDF file
	pdfauthor={\myauthor}		% - Set author of the PDF file
}
% ==========================================================================

% Header and footer
% ==========================================================================
\pagestyle{fancy}		% - Style of header/footer	
\lhead{Left header} % - Left header text
\chead{Center header} % - Center header text
\rhead{Right header} % - Right header text
% -------------------------------------------------------------------------
\lfoot{Left footer} % - Left footer text
\cfoot{Center footer - Page number \thepage} % - Center footer text
\rfoot{\thepage\ of \pageref{LastPage}} % - Right footer
% -------------------------------------------------------------------------
\renewcommand{\headrulewidth}{0.4pt} % - Black line below header text
\renewcommand{\footrulewidth}{0.4pt} % - Black line over footer text
% ==========================================================================

% Image resources
% ==========================================================================
\graphicspath{ {resources/} } % - Set directory which contains the images
% ==========================================================================

% Remove default indent
% ==========================================================================
%\setlength{\parindent}{0pt} % - Uncomment to remove all indents
% ==========================================================================

% Go deeper than a subsubsection (https://tex.stackexchange.com/a/60212)
% ==========================================================================
\titleclass{\subsubsubsection}{straight}[\subsection]
\newcounter{subsubsubsection}[subsubsection]
\renewcommand\thesubsubsubsection{\thesubsubsection.\arabic{subsubsubsection}}
\renewcommand\theparagraph{\thesubsubsubsection.\arabic{paragraph}} % optional; useful if paragraphs are to be numbered
\titleformat{\subsubsubsection}
{\normalfont\normalsize\bfseries}{\thesubsubsubsection}{1em}{}
\titlespacing*{\subsubsubsection}
{0pt}{3.25ex plus 1ex minus .2ex}{1.5ex plus .2ex}
\makeatletter
\renewcommand\paragraph{\@startsection{paragraph}{5}{\z@}%
	{3.25ex \@plus1ex \@minus.2ex}%
	{-1em}%
	{\normalfont\normalsize\bfseries}}
\renewcommand\subparagraph{\@startsection{subparagraph}{6}{\parindent}%
	{3.25ex \@plus1ex \@minus .2ex}%
	{-1em}%
	{\normalfont\normalsize\bfseries}}
\def\toclevel@subsubsubsection{4}
\def\toclevel@paragraph{5}
\def\toclevel@paragraph{6}
\def\l@subsubsubsection{\@dottedtocline{4}{7em}{4em}}
\def\l@paragraph{\@dottedtocline{5}{10em}{5em}}
\def\l@subparagraph{\@dottedtocline{6}{14em}{6em}}
\makeatother
\setcounter{secnumdepth}{4}
\setcounter{tocdepth}{4}
% ==========================================================================


% Document content begin
\begin{document}\thispagestyle{empty}

% Create title - changes the pagestyle
\maketitle

% Abstract of document
\begin{abstract}
\centering % - Center the contents on the page
Short summary of the document aka abstract.
\end{abstract}

% Show the table of contents - changes the pagestyle
\tableofcontents

% Use the following command to remove any headers and footers on the page at this point
\thispagestyle{fancy} % - Use "fancy" instead of "empty" if you want header/footer

% Use this when you use the document layout "\documentclass{report}"
% \chapter{First Chapter}

% Document section
\section{Introduction}

Text...

% Document section
\section{Section}

% Document section
\subsection{Subsection}

Text...

\paragraph{Paragraph}

Lorem ipsum dolor sit amet, consectetur adipiscing elit, sed do eiusmod tempor incididunt ut labore et dolore magna aliqua. Adipiscing diam donec adipiscing tristique risus nec feugiat.

New sentences that have a insertion on the right. Lorem ipsum dolor sit amet, consectetur adipiscing elit, sed do eiusmod tempor incididunt ut labore et dolore magna aliqua.
\newline
To not have this insertion/indent you need the command \verb|\newline|
\\
or \verb|\\| in the empty line.

To not have this insertion/indent you can also ban the indents globally by adding the line \verb|\setlength{\parindent}{0pt}|.

\vspace{5mm} %5mm vertical space

To get some space between texts use something like \verb|\vspace{5mm}| to get 5mm horizontal space.


% Document section
\subsection{Another subsection}

Text...

\subsubsection{Subsubsection}

Text...

\subsubsubsection{Subsubsubsection}

Text...

\paragraph{Paragraph}

Text...

\section{Text formatting}
\label{sec:exampleAnchor}

\subsection{Text types}

\begin{itemize}
	\item \textbf{Bold text} (\verb|\textbf{...}|)
	\item \underline{Underline text} (\verb|\underline{...}|)
	\item \textit{Italic text} (\verb|\textit{...}|)
	\item \textbf{\textit{\underline{Combine them}}} (\verb|\textbf{\textit{\underline{...}|)
	\item Normal text
	\item \textsl{Slanted style text} (\verb|\textsl{...}|)
	\item \textsc{Small caps style text} (\verb|\textsc{...}|)
	\item \MakeUppercase{uppeRcase text} (\verb|\MakeUppercase{...}|)
	\item \MakeLowercase{LOWERcASE tExT} (\verb|\MakeLowercase{...}|)
\end{itemize}

\subsection{Text color}

\begin{itemize}
	\item \textcolor{color-name}{Declared color text} (\verb|\textcolor{color-name}{...}|)
	\item \textcolor{red}{Red text} (\verb|\textcolor{red}{...}|)
	\item \textcolor{green}{Green text} (\verb|\textcolor{green}{...}|)
	\item \colorbox{black}{\color{white}{White text on black}} (\verb|\colorbox{black}{\color{white}{...}|)
	\item \fcolorbox{black}{yellow}{Text in black frame with yellow background} (\verb|\fcolorbox{black}{yellow}{...}|)
	\item \fcolorbox{red}{red}{\colorbox{cyan}{\color{white}{Ultimate combo}}} (\verb|\fcolorbox{red}{red}{\colorbox{cyan}{\color{white}{...}|)
\end{itemize}

\subsection{Text sizes}

\begin{itemize}
	\item {\tiny Tiny text} (\verb|{\tiny ...}|)
	\item {\scriptsize Scriptsize text} (\verb|{\scriptsize ...}|)
	\item {\footnotesize Footnotesize text} (\verb|{\footnotesize ...}|)
	\item {\small Small text} (\verb|{\small ...}|)
	\item {\normalsize Normalsize text} (\verb|{\normalsize ...}|)
	\item Normal text
	\item {\large large text} (\verb|{\large ...}|)
	\item {\Large Large text} (\verb|{\Large ...}|)
	\item {\LARGE LARGE text} (\verb|{\LARGE ...}|)
	\item {\huge huge text} (\verb|{\huge ...}|)
	\item {\Huge Huge text} (\verb|{\Huge ...}|)
\end{itemize}

\subsection{Text font}

\begin{itemize}
	\item Normal text
	\item \textrm{Serif (roman) text} (\verb|\textrm{...}|)
	\item \textsf{Sans serif text} (\verb|\textsf{...}|)
	\item \texttt{Typewriter (monospace) text} (\verb|\texttt{...}|)
\end{itemize}

\subsection{Text links}

\begin{itemize}
	\item \hyperref[sec:exampleAnchor]{Anchor link to a section}, Page number of reference: \pageref{sec:exampleAnchor} \\(\verb|\label{sec:...},\hyperref[sec:...]{...},\pageref{sec:...}|)
	\item \hyperref[fig:pixelGraphic]{Anchor link to a figure}, Page number of reference: \pageref{fig:pixelGraphic} \\(\verb|\label{fig:...},\hyperref[fig:...]{...},\pageref{fig:...}|)
	\item Reference\cite{source01,source02} (\verb|\cite{...}|)
	\item \url{https://www.wikibooks.org} (\verb|\url{https://...}|)
	\item \href{https://www.wikibooks.org}{Wikibooks home} (\verb|\href{https://...}{...}|)
\end{itemize}

\section{Lists}

\subsection{Unordered}

\paragraph{Simple list:}

\begin{itemize}
	\item One entry in the list
	\item Another entry in the list
\end{itemize}

\paragraph{Nested list:}

\begin{itemize}
	\item  First Level
	\begin{itemize}
		\item  Second Level
		\begin{itemize}
			\item  Third Level
			\begin{itemize}
				\item  Fourth Level
			\end{itemize}
		\end{itemize}
	\end{itemize}
\end{itemize}

\subsection{Ordered}

\paragraph{Simple list:}

\begin{enumerate}
	\item One entry in the list
	\item Another entry in the list
\end{enumerate}

\paragraph{Nested list:}

\begin{enumerate}
	%\renewcommand{\labelenumii}{\Roman{enumii}} % - Make ii roman number
	%\renewcommand{\labelenumii}{\{enumii}} % - Make ii roman number
	\item First level item
	\item First level item
	\begin{enumerate}
		\setcounter{enumii}{4} % - Begin at 5 (e)
		\item Second level item
		\item Second level item
		\begin{enumerate}
			\setcounter{enumiii}{1} % - Begin at 2 (ii)
			\item Third level item
			\item Third level item
			\begin{enumerate}
				\setcounter{enumiv}{9} % - Begin at 10 (J)
				\item Fourth level item
				\item Fourth level item
			\end{enumerate}
		\end{enumerate}
	\end{enumerate}
\end{enumerate}

\subsection{Text list}

\begin{description}[align=left]
	\item [One left] More text
	\item [Two left] More text
\end{description}

\begin{lstlisting}[language=tex,frame=single]
\begin{description}[align=left]
	\item [One left] More text
	\item [Two left] More text
\end{description}
\end{lstlisting}

\begin{description}[align=right,labelwidth=6cm]
	\item [One right + labelwidth] More text
	\item [Two right + labelwidth] More text
\end{description}

\begin{lstlisting}[language=tex,frame=single]
\begin{description}[align=right,labelwidth=3cm]
	\item [One right + labelwidth] More text
	\item [Two right + labelwidth] More text
\end{description}
\end{lstlisting}

\subsubsection{Combine them}

\begin{enumerate}
	\item The labels consists of sequential numbers. Hallöchenßß
	\begin{itemize}
		\item The individual entries are indicated with a black dot, a so-called bullet.
		\item The text in the entries may be of any length.
	\end{itemize}
	\item The numbers starts at 1 with every call to the enumerate environment.
	\begin{itemize}
		\item The individual entries are indicated with a black dot, a so-called bullet.
		\begin{itemize}
			\item The individual entries are indicated with a black dot, a so-called bullet.
			\begin{itemize}
				\item The individual entries are indicated with a black dot, a so-called bullet.
			\end{itemize}
		\end{itemize}
	\end{itemize}
\end{enumerate}

\section{Equations}

Inside of text you can use the inline formatting to show equations or symbols by inserting it inside two \verb|$| symbols like this: $E=mc^2$ (\verb|$E=mc^2$|).

\vspace{5mm}

Also in this document we can define something by using the two \verb|$| symbols instead of one and writing the point of the sentence at the end of the formula (\verb|$$E=mc^2.$$|)  $$E=mc^2.$$

\vspace{5mm}

There is also the option to show big equations with this (\verb|$$E=mc^2$$|):

$$x^n+y^n=z^n$$

But the better way is this:

\begin{equation}
\label{simple_equation}
\alpha = \sqrt{ \beta }
\end{equation}

\begin{equation}
\label{simple_equation}
\alpha = \sqrt{ \beta }
\end{equation}

% Create a new page
\newpage

\section{Tables}

\subsection{Width specified for a column}

\begin{center}
	\begin{tabular}{ | l | l | l | p{5cm} |}
		\hline
		Day & Min Temp & Max Temp & Summary \\ \hline
		Monday & 11C & 22C & A clear day with lots of sunshine.  
		However, the strong breeze will bring down the temperatures. \\ \hline
		Tuesday & 9C & 19C & Cloudy with rain, across many northern regions. Clear spells 
		across most of Scotland and Northern Ireland, 
		but rain reaching the far northwest. \\ \hline
		Wednesday & 10C & 21C & Rain will still linger for the morning. 
		Conditions will improve by early afternoon and continue 
		throughout the evening. \\
		\hline
	\end{tabular}
\end{center}

\vspace{5mm}

\begin{tabular}{l*{6}{c}r}
	Team              & P & W & D & L & F  & A & Pts \\
	\hline
	Manchester United & 6 & 4 & 0 & 2 & 10 & 5 & 12  \\
	Celtic            & 6 & 3 & 0 & 3 &  8 & 9 &  9  \\
	Benfica           & 6 & 2 & 1 & 3 &  7 & 8 &  7  \\
	FC Copenhagen     & 6 & 2 & 1 & 3 &  5 & 8 &  7  \\
\end{tabular}

\vspace{5mm}

\newcolumntype{d}[1]{D{.}{\cdot}{#1} } % Table sort decimal

\begin{tabular}{l r c d{1} } % For decimal sorting look at the top
	Left&Right&Center&\mathrm{Decimal}\\
	1&2&3&4\\
	11&22&33&44\\
	1.1&2.2&3.3&4.4\\
\end{tabular}

\vspace{5mm}

\begin{tabular}{ |l|l| }
	\hline
	\multicolumn{2}{|c|}{Team sheet} \\
	\hline
	GK & Paul Robinson \\
	LB & Lucas Radebe \\
	DC & Michael Duberry \\
	DC & Dominic Matteo \\
	RB & Dider Domi \\
	MC & David Batty \\
	MC & Eirik Bakke \\
	MC & Jody Morris \\
	FW & Jamie McMaster \\
	ST & Alan Smith \\
	ST & Mark Viduka \\
	\hline
\end{tabular}

\vspace{5mm}

\begin{tabular}{cc|c|c|c|c|l} % \usepackage{multirow}
	\cline{3-6}
	& & \multicolumn{4}{ c| }{Primes} \\ \cline{3-6}
	& & 2 & 3 & 5 & 7 \\ \cline{1-6}
	\multicolumn{1}{ |c  }{\multirow{2}{*}{Powers} } &
	\multicolumn{1}{ |c| }{504} & 3 & 2 & 0 & 1 &     \\ \cline{2-6}
	\multicolumn{1}{ |c  }{}                        &
	\multicolumn{1}{ |c| }{540} & 2 & 3 & 1 & 0 &     \\ \cline{1-6}
	\multicolumn{1}{ |c  }{\multirow{2}{*}{Powers} } &
	\multicolumn{1}{ |c| }{gcd} & 2 & 2 & 0 & 0 & min \\ \cline{2-6}
	\multicolumn{1}{ |c  }{}                        &
	\multicolumn{1}{ |c| }{lcm} & 3 & 3 & 1 & 1 & max \\ \cline{1-6}
\end{tabular}

\vspace{5mm}

\begin{tabular}{ r|c|c| }
	\multicolumn{1}{r}{}
	&  \multicolumn{1}{c}{noninteractive}
	& \multicolumn{1}{c}{interactive} \\
	\cline{2-3}
	massively multiple & Library & University \\
	\cline{2-3}
	one-to-one & Book & Tutor \\
	\cline{2-3}
\end{tabular}

\vspace{5mm}

\begin{tabular*}{0.75\textwidth}{@{\extracolsep{\fill} } | c | c | c | r | }
	\hline
	label 1 & label 2 & label 3 & label 4 \\
	\hline 
	item 1  & item 2  & item 3  & item 4  \\
	\hline
\end{tabular*}

\vspace{5mm}

\begin{tabular}{llr} % use package "array" or "dcolumn"
	\firsthline
	\multicolumn{2}{c}{Item} \\
	\cline{1-2}
	Animal    & Description & Price (\$) \\
	\hline
	Gnat      & per gram    & 13.65      \\
	& each        & 0.01       \\
	Gnu       & stuffed     & 92.50      \\
	Emu       & stuffed     & 33.33      \\
	Armadillo & frozen      & 8.99       \\
	\lasthline
\end{tabular}

\vspace{5mm}

\begin{tabular}{r@{.}l}
	3   & 14159 \\
	16  & 2     \\
	123 & 456   \\
\end{tabular}

\newpage

\section{Embed content}

\subsection{Text/Code files}

\lstinputlisting[frame=single]{resources/example-resource-python.py}

\subsection{Image files}

\begin{figure}[ht]
	\centering
	\includegraphics[width=0.5\textwidth]{example-resource-pixel-graphic.png}
	\caption{Pixel graphic}
	\label{fig:pixelGraphic}
\end{figure}

\begin{figure}[ht]
	\centering
	\includegraphics[width=0.5\textwidth]{example-resource-pdf-file.pdf}
	\caption{PDF file page}
	\label{fig:pdfFile}
\end{figure}

\vfill

\section{Heading at the end of the page}

Thanks to the command \verb|\vfill| a new vertical space will be created and push this section as low as it can get.

\newpage

\begin{thebibliography}{3} % - Number of entries to be added -> cannot be higher than 99
	
	\bibitem{source01}
	Author One, Author Two.
	\textit{The document}. 
	University, Place, State/Country, Year.
	
	\bibitem{source02} 
	Author One, Author Two.
	\textit{The document}. 
	University, Place, State/Country, Year.
	
	\bibitem{source03}
	Author: \textit{The thing}. 
	\\\url{https://www.wikibooks.org}
	
\end{thebibliography}

% Document end
\end{document}

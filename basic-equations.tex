% This document was created while following the amzaing LaTeX beginner tutorial (https://www.youtube.com/playlist?list=PL542920k_cOoSTC23aG3jhczwHtrRKb9q) of the YouTube channel DorFuchs (https://www.youtube.com/channel/UC97dp7op_ZjSCNp3DRLGymQ)

\documentclass{article}


% packages to import:
\usepackage[ngerman]{babel} % Language package for usign for example umlauts
\usepackage[utf8]{inputenc} % Input encoding - rendering of for example umlauts
\usepackage[T1]{fontenc} % Font encoding to find later words with for example umlauts
\usepackage{amsmath} % Use advanced math in \begin{align}\end{align} sections
\usepackage{amssymb} % Use math symbols like the symbol for the real numbers
\usepackage{amsthm} % Create your own sections and use for example a proof section


% create a section named satz
\newtheorem{satz}{Satz}

% create custom commands ([#] is the number of arguments)
\newcommand{\R}{\mathbb{R}}
\newcommand{\vektor}[1]{\begin{pmatrix}#1\end{pmatrix}}

% remove indent
\setlength{\parindent}{0px}


\begin{document}
	

\title{Testdokument}
\author{DorFuchs}
\maketitle
	

% add table of contents to the document
\tableofcontents

% show text on the next side
\newpage


\section{Satz des Pythagoras}
	
\begin{satz}
	In einem rechtwinkligen Dreieck mit Katheten $a$ und $b$ und Hyphotenuse $c$ gilt
	\[ a^2 + b^2 = c^2 . \]
\end{satz}
\begin{proof}
	... \\
	... \\
	...
\end{proof}


\section{p-q-Formel}

Für die Gleichung $x^2 + px + q = 0$, wobei $p, q \in \mathbb{R}$ mit $\frac{p}{2}^2 - q > 0$, lautet die Lösungsformel
\begin{align}
	x &= \frac{p}{2} \pm \sqrt{ \left( \frac{p}{2} \right)^2 - q} \\
	&= - \frac{p}{2} \pm \sqrt{\frac{p^2}{4} - q}
\end{align}


\section{Euklidische Geometrie}

\subsection{Matrizen}

\begin{satz}
	Die Einheitsmatrix im $\R^3$ lautet
	\[ \begin{pmatrix}
	1 & 0 & 0 \\
	0 & 1 & 0 \\
	0 & 0 & 1
	\end{pmatrix} . \]
\end{satz}

\subsection{Vektoren}

\begin{satz}
	Der Betrag eines Vektors ist im $\R^3$ ist definiert durch
	\[ \left| \vektor{x\\y\\z} \right| = \sqrt{x^2 + y^2 + z^2} . \]
\end{satz}


\end{document}
% Document type/layout
% ==========================================================================
\documentclass{article}				% -> Article layout (default)
%\documentclass[journal]{IEEEtran}	% -> Two column journal layout
%\documentclass{report}				% -> Report layout
% ==========================================================================

% Add packages to expand the functionality with the following command:
% \usepackage[options]{packagename}
% ==========================================================================
\usepackage{bookmark}	% - Show all sections in the pdf as metadata
\usepackage{hyperref}	% - For metada rendering in PDFs
\usepackage{lastpage}	% - To get the toatal numbe rof pages
\usepackage{fancyhdr}	% - For header and footer
\usepackage{tocloft}	% - For table of contents
\usepackage{xcolor}		% - Use colors
\usepackage{graphicx}	% - Load/Display graphics
\usepackage{hyperref}	% - Insert hyperlinks
\usepackage{enumitem}	% - Needed for text lists
\usepackage{listings}	% - Display source code
% ==========================================================================

% Variables
% ==========================================================================
\newcommand*{\mytitle}{The title}		% - Document title
\newcommand*{\myauthor}{One author}		% - Document author (one author)
% \newcommand*{\myauthorOne}{AuthorOne}	% - Document authors (many authors)
% \newcommand*{\myauthorTwo}{AuthorTwo}	% - Document authors (many authors)
% \newcommand*{\myauthor}{\texorpdfstring{\myauthorOne, \myauthorTwo}{\myauthorOne;\myauthorTwo}} % - Document authors (many authors)
\newcommand*{\mysubject}{The subject}			% - Document subject
\newcommand*{\mydate}{\normalsize 03.09.2018}	% - Document release date
% -------------------------------------------------------------------------
\definecolor{color-name}{HTML}{FF7F00}
\definecolor{orange}{RGB}{255,127,0}
\definecolor{orange}{HTML}{FF7F00}
% ==========================================================================

% Document data
% ==========================================================================
\title{\mytitle}	% - Set the visual title
\author{\myauthor}	% - Set the visual author
\date{\mydate}		% - Set the visual date
\hypersetup{		% - Metadata setup
	pdftitle={\mytitle},		% - Set title of the PDF file
	pdfsubject={\mysubject},	% - Set subject of the PDF file
	pdfauthor={\myauthor}		% - Set author of the PDF file
}
% ==========================================================================

% Header and footer
% ==========================================================================
\pagestyle{fancy}		% - Style of header/footer	
\lhead{Left header} % - Left header text
\chead{Center header} % - Center header text
\rhead{Right header} % - Right header text
% -------------------------------------------------------------------------
\lfoot{Left footer} % - Left footer text
\cfoot{Center footer - Page number \thepage} % - Center footer text
\rfoot{\thepage\ of \pageref{LastPage}} % - Right footer
% -------------------------------------------------------------------------
\renewcommand{\headrulewidth}{0.4pt} % - Black line below header text
\renewcommand{\footrulewidth}{0.4pt} % - Black line over footer text
% ==========================================================================

% Image resources
% ==========================================================================
\graphicspath{ {resources/} } % - Set directory which contains the images
% ==========================================================================


% Document content begin
\begin{document}\thispagestyle{empty}

% Create title - changes the pagestyle
\maketitle

% Abstract of document
\begin{abstract}
\centering % - Center the contents on the page
Short summary of the document aka abstract.
\end{abstract}

% Show the table of contents - changes the pagestyle
\tableofcontents

% Use the following command to remove any headers and footers on the page at this point
\thispagestyle{fancy} % - Use "fancy" instead of "empty" if you want header/footer

% Use this when you use the document layout "\documentclass{report}"
% \chapter{First Chapter}

% Document section
\section{Introduction}

Einführung...

% Document section
\section{Section}

% Document section
\subsection{Subsection}

Text...

% Document section
\subsection{Another subsection}

Text...

\subsubsection{Subsubsection}

Text...

\section{Lists}

\subsection{Unordered}

\paragraph{Simple list:}

\begin{itemize}
	\item One entry in the list
	\item Another entry in the list
\end{itemize}

\paragraph{Nested list:}

\begin{itemize}
	\item  First Level
	\begin{itemize}
		\item  Second Level
		\begin{itemize}
			\item  Third Level
			\begin{itemize}
				\item  Fourth Level
			\end{itemize}
		\end{itemize}
	\end{itemize}
\end{itemize}

\subsection{Ordered}

\paragraph{Simple list:}

\begin{enumerate}
	\item One entry in the list
	\item Another entry in the list
\end{enumerate}

\paragraph{Nested list:}

\begin{enumerate}
	%\renewcommand{\labelenumii}{\Roman{enumii}} % - Make ii roman number
	%\renewcommand{\labelenumii}{\{enumii}} % - Make ii roman number
	\item First level item
	\item First level item
	\begin{enumerate}
		\setcounter{enumii}{4} % - Begin at 5 (e)
		\item Second level item
		\item Second level item
		\begin{enumerate}
			\setcounter{enumiii}{1} % - Begin at 2 (ii)
			\item Third level item
			\item Third level item
			\begin{enumerate}
				\setcounter{enumiv}{9} % - Begin at 10 (J)
				\item Fourth level item
				\item Fourth level item
			\end{enumerate}
		\end{enumerate}
	\end{enumerate}
\end{enumerate}

\subsection{Text list}

\begin{description}[align=left]
	\item [One] More text
	\item [Two] More text
\end{description}

\begin{lstlisting}[language=tex,frame=single]
\begin{description}[align=left]
	\item [One] More text
	\item [Two] More text
\end{description}
\end{lstlisting}

\begin{description}[align=right]
	\item [One right] More text
	\item [Two right] More text
\end{description}

\begin{lstlisting}[language=tex,frame=single]
\begin{description}[align=right]
	\item [One] More text
	\item [Two] More text
\end{description}
\end{lstlisting}

\begin{description}[align=right,labelwidth=3cm]
	\item [One right] More text
	\item [Two right] More text
\end{description}

\begin{lstlisting}[language=tex,frame=single]
\begin{description}[align=right,labelwidth=3cm]
	\item [One] More text
	\item [Two] More text
\end{description}
\end{lstlisting}

\subsubsection{Combine them}

\begin{enumerate}
	\item The labels consists of sequential numbers.
	\begin{itemize}
		\item The individual entries are indicated with a black dot, a so-called bullet.
		\item The text in the entries may be of any length.
	\end{itemize}
	\item The numbers starts at 1 with every call to the enumerate environment.
	\begin{itemize}
		\item The individual entries are indicated with a black dot, a so-called bullet.
		\begin{itemize}
			\item The individual entries are indicated with a black dot, a so-called bullet.
			\begin{itemize}
				\item The individual entries are indicated with a black dot, a so-called bullet.
			\end{itemize}
		\end{itemize}
	\end{itemize}
\end{enumerate}

\subsubsection{Subsubsection}
A normal sentence.

A new sentence with insertion.
\newline
Thanks to the \verb|\newline| command this insertions goes away.\\
To block concatenating on the next line you use \verb|\\| at the end of the line.

\section{Equations}

Inside of text you can use the inline formatting to show equations or symbols by inserting it inside two  \verb|$| symbols like this: $E=mc^2$ (\verb|$E=mc^2$|).

\vspace{5mm} %5mm vertical space

Or show big equations like this (\verb|$$E=mc^2$$|):

$$x^n+y^n=z^n$$

Or show big equations like this:

\begin{equation}
\label{simple_equation}
\alpha = \sqrt{ \beta }
\end{equation}


\begin{figure}
    \centering
%    \includegraphics[width=3.0in]{myfigure}
    \caption{Simulation Results}
    \label{simulationfigure}
\end{figure}

% Create a new page
\newpage

\section{Text formatting}
\label{sec:exampleAnchor}

\subsection{Text types}

\begin{itemize}
	\item \textbf{Bold text} (\verb|\textbf{...}|)
	\item \underline{Underline text} (\verb|\underline{...}|)
	\item \textit{Italic text} (\verb|\textit{...}|)
	\item \textbf{\textit{\underline{Combine them}}} (\verb|\textbf{\textit{\underline{...}|)
	\item Normal text
	\item \textsl{Slanted style text} (\verb|\textsl{...}|)
	\item \textsc{Small caps style text} (\verb|\textsc{...}|)
	\item \MakeUppercase{uppeRcase text} (\verb|\MakeUppercase{...}|)
	\item \MakeLowercase{LOWERcASE tExT} (\verb|\MakeLowercase{...}|)
\end{itemize}

\subsection{Text color}

\begin{itemize}
	\item \textcolor{color-name}{Declared color text} (\verb|\textcolor{color-name}{...}|)
	\item \textcolor{red}{Red text} (\verb|\textcolor{red}{...}|)
	\item \textcolor{green}{Green text} (\verb|\textcolor{green}{...}|)
	\item \colorbox{black}{\color{white}{White text on black}} (\verb|\colorbox{black}{\color{white}{...}|)
	\item \fcolorbox{black}{yellow}{Text in black frame with yellow background} (\verb|\fcolorbox{black}{yellow}{...}|)
	\item \fcolorbox{red}{red}{\colorbox{cyan}{\color{white}{Ultimate combo}}} (\verb|\fcolorbox{red}{red}{\colorbox{cyan}{\color{white}{...}|)
\end{itemize}

\subsection{Text sizes}

\begin{itemize}
	\item {\tiny Tiny text} (\verb|{\tiny ...}|)
	\item {\scriptsize Scriptsize text} (\verb|{\scriptsize ...}|)
	\item {\footnotesize Footnotesize text} (\verb|{\footnotesize ...}|)
	\item {\small Small text} (\verb|{\small ...}|)
	\item {\normalsize Normalsize text} (\verb|{\normalsize ...}|)
	\item Normal text
	\item {\large large text} (\verb|{\large ...}|)
	\item {\Large Large text} (\verb|{\Large ...}|)
	\item {\LARGE LARGE text} (\verb|{\LARGE ...}|)
	\item {\huge huge text} (\verb|{\huge ...}|)
	\item {\Huge Huge text} (\verb|{\Huge ...}|)
\end{itemize}

\subsection{Text font}

\begin{itemize}
	\item Normal text
	\item \textrm{Serif (roman) text} (\verb|\textrm{...}|)
	\item \textsf{Sans serif text} (\verb|\textsf{...}|)
	\item \texttt{Typewriter (monospace) text} (\verb|\texttt{...}|)
\end{itemize}

\subsection{Text links}

\begin{itemize}
	\item \hyperref[sec:exampleAnchor]{Anchor link to a section}, Page number of reference: \pageref{sec:exampleAnchor} \\(\verb|\label{sec:...},\hyperref[sec:...]{...},\pageref{sec:...}|)
	\item \hyperref[fig:pixelGraphic]{Anchor link to a figure}, Page number of reference: \pageref{fig:pixelGraphic} \\(\verb|\label{fig:...},\hyperref[fig:...]{...},\pageref{fig:...}|)
	\item Reference\cite{source01,source02} (\verb|\cite{...}|)
	\item \url{https://www.wikibooks.org} (\verb|\url{https://...}|)
	\item \href{https://www.wikibooks.org}{Wikibooks home} (\verb|\href{https://...}{...}|)
\end{itemize}

\section{Embed content}

\subsection{Text/Code files}

\subsection{Image files}

\begin{figure}[ht]
	\centering
	\includegraphics[width=0.5\textwidth]{example-resource-pixel-graphic.png}
	\caption{Pixel graphic}
	\label{fig:pixelGraphic}
\end{figure}

\begin{figure}[ht]
	\centering
	\includegraphics[width=0.5\textwidth]{example-resource-pdf-file.pdf}
	\caption{PDF file page}
	\label{fig:pdfFile}
\end{figure}

\vfill

\section{Heading at the end of the page}

Thanks to the command \verb|\vfill| a new vertical space will be created and push this section as low as it can get.

\newpage

\begin{thebibliography}{3} % - Number of entries to be added -> cannot be higher than 99
	
	\bibitem{source01}
	Author One, Author Two.
	\textit{The document}. 
	University, Place, State/Country, Year.
	
	\bibitem{source02} 
	Author One, Author Two.
	\textit{The document}. 
	University, Place, State/Country, Year.
	
	\bibitem{source03}
	Author: \textit{The thing}. 
	\\\url{https://www.wikibooks.org}
	
\end{thebibliography}

% Document end
\end{document}

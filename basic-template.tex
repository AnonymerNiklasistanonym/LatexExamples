% Document type/layout
% ==========================================================================
\documentclass{article}				% -> Article layout (default)
%\documentclass[journal]{IEEEtran}	% -> Two column journal layout
%\documentclass{report}				% -> Report layout
% ==========================================================================

% Add packages to expand the functionality with the following command:
% \usepackage[options]{packagename}
% ==========================================================================
\usepackage{bookmark}	% - Show all sections in the pdf as metadata
\usepackage{hyperref}	% - For metada rendering in PDFs
\usepackage{lastpage}	% - To get the toatal numbe rof pages
\usepackage{fancyhdr}	% - For header and footer
\usepackage{tocloft}	% - For table of contents
% ==========================================================================

% Variables
% ==========================================================================
\newcommand*{\mytitle}{The title}		% - Document title
\newcommand*{\myauthor}{One author}		% - Document author (one author)
% \newcommand*{\myauthorOne}{AuthorOne}	% - Document authors (many authors)
% \newcommand*{\myauthorTwo}{AuthorTwo}	% - Document authors (many authors)
% \newcommand*{\myauthor}{\texorpdfstring{\myauthorOne, \myauthorTwo}{\myauthorOne;\myauthorTwo}} % - Document authors (many authors)
\newcommand*{\mysubject}{The subject}			% - Document subject
\newcommand*{\mydate}{\normalsize 03.09.2018}	% - Document release date
% ==========================================================================

% Document data
% ==========================================================================
\title{\mytitle}	% - Set the visual title
\author{\myauthor}	% - Set the visual author
\date{\mydate}		% - Set the visual date
\hypersetup{		% - Metadata setup
	pdftitle={\mytitle},		% - Set title of the PDF file
	pdfsubject={\mysubject},	% - Set subject of the PDF file
	pdfauthor={\myauthor}		% - Set author of the PDF file
}
% ==========================================================================

% Header and footer
% ==========================================================================
\pagestyle{fancy}		% - Style of header/footer	
\lhead{Left header} % - Left header text
\chead{Center header} % - Center header text
\rhead{Right header} % - Right header text
% -------------------------------------------------------------------------
\lfoot{Left footer} % - Left footer text
\cfoot{Center footer - Page number \thepage} % - Center footer text
\rfoot{\thepage\ of \pageref{LastPage}} % - Right footer
% -------------------------------------------------------------------------
\renewcommand{\headrulewidth}{0.4pt} % - Black line below header text
\renewcommand{\footrulewidth}{0.4pt} % - Black line over footer text
% ==========================================================================


% Document content begin
\begin{document}\thispagestyle{empty}

% Create title - changes the pagestyle
\maketitle

% Abstract of document
\begin{abstract}
\centering % - Center the contents on the page
Short summary of the document aka abstract.
\end{abstract}

% Show the table of contents - changes the pagestyle
\tableofcontents

% Use the following command to remove any headers and footers on the page at this point
\thispagestyle{fancy} % - Use "fancy" instead of "empty" if you want header/footer

% Use this when you use the document layout "\documentclass{report}"
% \chapter{First Chapter}

% Document section
\section{Introduction}

Lorem ipsum dolor sit amet, consectetur adipisicing elit, sed do eiusmod tempor
incididunt ut labore et dolore magna aliqua. Ut enim ad minim veniam, quis
nostrud exercitation ullamco laboris nisi ut aliquip ex ea commodo consequat.
Duis aute irure dolor in reprehenderit in voluptate velit esse cillum dolore eu
fugiat nulla pariatur. Excepteur sint occaecat cupidatat non proident, sunt in
culpa qui officia deserunt mollit anim id est laborum.

% Document section
\section{Section}

% Document section
\subsection{Subsection}

Lorem ipsum dolor sit amet, consectetur adipisicing elit, sed do eiusmod tempor
incididunt ut labore et dolore magna aliqua. Ut enim ad minim veniam, quis
nostrud exercitation ullamco laboris nisi ut aliquip ex ea commodo consequat.
Duis aute irure dolor in reprehenderit in voluptate velit esse cillum dolore eu
fugiat nulla pariatur. Excepteur sint occaecat cupidatat non proident, sunt in
culpa qui officia deserunt mollit anim id est laborum.

% Document section
\subsection{Another subsection}

\subsubsection{Subsubsection}

test

\subsection{Lists}

\subsubsection{Unordered}

Simple list:

\begin{itemize}
	\item One entry in the list
	\item Another entry in the list
\end{itemize}

Nested list:

\begin{itemize}
	\item  First Level
	\begin{itemize}
		\item  Second Level
		\begin{itemize}
			\item  Third Level
			\begin{itemize}
				\item  Fourth Level
			\end{itemize}
		\end{itemize}
	\end{itemize}
\end{itemize}

\subsubsection{Ordered}

Simple list:

\begin{enumerate}
	\item One entry in the list
	\item Another entry in the list
\end{enumerate}

Nested list:

\renewcommand{\labelenumii}{\Roman{enumii}}
\begin{enumerate}
	\item First level item
	\item First level item
	\begin{enumerate}
		\setcounter{enumii}{4}
		\item Second level item
		\item Second level item
		\begin{enumerate}
			\item Third level item
			\item Third level item
			\begin{enumerate}
				\item Fourth level item
				\item Fourth level item
			\end{enumerate}
		\end{enumerate}
	\end{enumerate}
\end{enumerate}

\subsubsection{Combine them}

\begin{enumerate}
	\item The labels consists of sequential numbers.
	\begin{itemize}
		\item The individual entries are indicated with a black dot, a so-called bullet.
		\item The text in the entries may be of any length.
	\end{itemize}
	\item The numbers starts at 1 with every call to the enumerate environment.
	\begin{itemize}
		\item The individual entries are indicated with a black dot, a so-called bullet.
		\begin{itemize}
			\item The individual entries are indicated with a black dot, a so-called bullet.
			\begin{itemize}
				\item The individual entries are indicated with a black dot, a so-called bullet.
			\end{itemize}
		\end{itemize}
	\end{itemize}
\end{enumerate}

\subsubsection{Subsubsection}
A normal sentence.

A new sentence with insertion.
\newline
Thanks to the \verb|\newline| command this insertions goes away.\\
To block concatenating on the next line you use \verb|\\| at the end of the line.

\section{Equations}

Inside of text you can use the inline formatting to show equations or symbols by inserting it inside two  \verb|$| symbols like this: $E=mc^2$ (\verb|$E=mc^2$|).

\vspace{5mm} %5mm vertical space

Or show big equations like this (\verb|$$E=mc^2$$|):

$$x^n+y^n=z^n$$

Or show big equations like this:

\begin{equation}
\label{simple_equation}
\alpha = \sqrt{ \beta }
\end{equation}


\begin{figure}
    \centering
%    \includegraphics[width=3.0in]{myfigure}
    \caption{Simulation Results}
    \label{simulationfigure}
\end{figure}

% Create a new page
\newpage

\section{Conclusion}

\subsection{Conclusion 2}

Lorem ipsum dolor sit amet, consectetur adipisicing elit, sed do eiusmod tempor
incididunt ut labore et dolore magna aliqua. Ut enim ad minim veniam, quis
nostrud exercitation ullamco laboris nisi ut aliquip ex ea commodo consequat.
Duis aute irure dolor in reprehenderit in voluptate velit esse cillum dolore eu
fugiat nulla pariatur.

\vfill

\section{Heading at the end of the page}

Thanks to the command \verb|\vfill| a new vertical space will be created and push this section as low as it can get.

% Document end
\end{document}
